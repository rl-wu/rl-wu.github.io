\documentclass[12pt,a4paper]{article}

% 中文支持
\usepackage[UTF8]{ctex}

% 数学宏包
\usepackage{amsmath,amsfonts,amssymb,blkarray,mathrsfs,amsthm}

%插入代码
\usepackage{listings}
\usepackage[most]{tcolorbox}

% 图片和表格
\usepackage{graphicx}
\usepackage{float}
\usepackage{booktabs}
\usepackage{subcaption}

% 页面布局
\usepackage{geometry}
\geometry{left=2.5cm,right=2.5cm,top=3cm,bottom=3cm}
\usepackage{multirow}


% 超链接
\usepackage{hyperref}
\hypersetup{
    colorlinks=true,
    linkcolor=black,
    citecolor=black,
    urlcolor=red
}


% 页眉和页脚
\usepackage{fancyhdr}
\pagestyle{fancy}
\fancyhf{}
\rhead{泛函分析笔记}
\lhead{WRL}
\rfoot{第 \thepage 页}

% 颜色
\usepackage{xcolor}


% 定义定理环境
\newtheorem{thm}{定理}[subsection]  % 在每个章节重新编号
\newtheorem{lemma}{引理}[subsection]    % 在每个章节重新编号
\newtheorem{corollary}{推论}[subsection] % 在每个章节重新编号
\newtheorem{definition}{定义}[subsection] % 在每个章节内自动编号



\begin{document}
\begin{center}
\section*{摘要}
\end{center}

本来寒假打算在个人主页上面更新泛函分析的自学笔记,但在markdown文件中插入latex真不太方便,
故单独用\LaTeX文件来记泛函笔记。

我自学使用的教材为张恭庆老师的《泛函分析讲义(上)》,在此笔记中主要记录书中的核心内容,配以心得体会。


{\centering\tableofcontents}

\newpage
\section{度量空间}
\subsection{压缩映射原理}
\subsection{完备化}
\subsection{列紧集}
\subsection{赋范线性空间}
\subsection{凸集与不动点}
\subsection{内积空间}

\begin{thm}
    如果$C$是\rm{Hilbert}空间$\mathscr{X}$中的一个闭凸子集,
    那么在$C$上存在唯一元素$x_0$取到最小范数.
\end{thm}
\begin{proof}
    存在性: 设$d= \inf_{z \in C}\|z\|$,取$x_n$,使得$d\leq \|x_n\|\leq d + \frac{1}{n}$利用
    \[\|x_m - x_n\|^2 = 2(\|x_m\|^2+\|x_n\|^2)-4\|\frac{x_m + x_n}{2}\|^2\]
    可以证明$\{x_n\}$是柯西列.\\
    唯一性: 同上利用极化恒等式可证唯一性.
\end{proof}
\begin{corollary}[\rm{Hilbert}空间闭凸子集中最佳逼近元存在且唯一]
    若C是\rm{Hilbert}空间$\mathscr{X}$中的一个闭凸子集,则对
    $\forall y \in \mathscr{X},\exists !x_0 \in C$,使得\[\|y-x_0\| = \inf_{x\in C}\|x-y\|\]
\end{corollary}
\begin{proof}
    将$C$平移$-y$之后,利用上面的定理即可.
\end{proof}
\begin{thm}[最佳逼近元的刻画]
    设$C$是内积空间$\mathscr{X}$中的闭凸子集, $\forall y \in \mathscr{X}$为了$x_0$是$y$在$C$上的最佳逼近元,必须且仅须它适合
    \[\mathrm{Re}(y-x_0,x_0-x)\geq 0,\forall x \in C\]
\end{thm}
\begin{proof}
    $\forall x \in C$,考虑函数$\varphi_x(t) = \|y-tx-(1-t)x_0\|^2$,为了$x_0$是$y$在$C$上的最佳逼近元,必须且仅需
    \[\varphi_x(t)\geq\varphi_x(0) \quad \forall x \in C ,\forall t \in[0,1]\]
    考察此二次函数的取值即可,这是容易地.
\end{proof}
\begin{corollary}
    设$M$是\rm{Hilbert}空间$\mathscr{X}$的一个闭线性子流形,$\forall x \in \mathscr{X}$,为了$y$是$x$在$M$上的最佳逼近元,必须且仅须它适合
    \[x-y \perp M-\{y\} \]
\end{corollary}
\begin{corollary}[正交分解]
    设$M$是\rm{Hilbert}空间$\mathscr{X}$中的一个闭线性子空间,那么$\forall x \in \mathscr{X}$,存在下列唯一的正交分解:
    \[x= y+z\quad (y\in M,z\in M^{\top})\]
\end{corollary}
\begin{proof}
    实际上,$y$为$x$在$M$上的最佳逼近元.
\end{proof}

\newpage
\section{线性算子与线性泛函}
\subsection{线性算子的概念}
\begin{definition}[线性算子]
\end{definition}
\begin{definition}[连续性与有界性]
\end{definition}
\subsection{Riesz表示定理及其应用}
\begin{thm}[Riesz表示定理(Hilbert空间)]
\end{thm}
Riesz表示定理有很多应用,这首先依赖于\rm{Hilbert}空间本身的良好性质,其次要求是连续线性泛函.
\begin{definition}[弱解]
\end{definition}
\begin{thm}[Laplace方程Dirichlet边值问题的弱解存在唯一]
\end{thm}
\subsection{纲与开映射定理}
\begin{definition}[疏集,第一纲集,第二纲集]
\end{definition}
\begin{thm}[Baire]
\end{thm}
\begin{thm}[开映射定理]
\end{thm}
\begin{thm}[Banach]
\end{thm}
\begin{definition}[闭线性算子]
\end{definition}
\begin{corollary}[等价范数定理]
\end{corollary}
\begin{thm}[闭图像定理]
\end{thm}
\begin{thm}[一致有界定理]
\end{thm}
\subsection{Hahn-Banach定理}
\subsection{共轭空间、弱收敛、自反空间}
\subsection{线性算子的谱}

\end{document}