\documentclass[12pt,a4paper]{article}

% 中文支持
\usepackage[UTF8]{ctex}

% 数学宏包
\usepackage{amsmath,amsfonts,amssymb,blkarray,mathrsfs,amsthm}

%插入代码
\usepackage{listings}
\usepackage[most]{tcolorbox}

% 图片和表格
\usepackage{graphicx}
\usepackage{float}
\usepackage{booktabs}
\usepackage{subcaption}

%使用分条的列表
\usepackage{enumitem}

% 页面布局
\usepackage{geometry}
\geometry{left=2.5cm,right=2.5cm,top=3cm,bottom=3cm}
\usepackage{multirow}


% 超链接
\usepackage{hyperref}
\hypersetup{
    colorlinks=true,
    linkcolor=black,
    citecolor=black,
    urlcolor=red
}


% 页眉和页脚
\usepackage{fancyhdr}
\pagestyle{fancy}
\fancyhf{}
\rhead{泛函分析笔记}
\lhead{WRL}
\rfoot{第 \thepage 页}

% 颜色
\usepackage{xcolor}


% 定义定理环境
\newtheorem{thm}{定理}[subsection]  % 在每个章节重新编号
\newtheorem{lemma}{引理}[subsection]    % 在每个章节重新编号
\newtheorem{corollary}{推论}[subsection] % 在每个章节重新编号
\newtheorem{definition}{定义}[subsection] % 在每个章节内自动编号
\newtheorem{example}{例}[subsection] % 在每个章节内自动编号




\begin{document}
\begin{center}
\section*{摘要}
\end{center}

本来寒假打算在个人主页上面更新泛函分析的自学笔记,但在markdown文件中插入latex真不太方便,
故单独用\LaTeX文件来记泛函笔记。

我自学使用的教材为张恭庆老师的《泛函分析讲义(上)》,在此笔记中主要记录书中的核心内容,配以心得体会。


{\centering\tableofcontents}

\newpage
\section{度量空间}
\subsection{压缩映射原理}
本节首先定义了\textbf{度量空间}.

\begin{example}[{空间 $C[a,b]$ }]
    区间 $[a,b]$ 上的连续函数全体记为 $C[a,b]$, 按距离 \[\rho(x,y) \triangleq \underset{a \leq t \leq b}{\max} | x(t) - y(t) | \]
    形成度量空间 $(C[a,b], \rho)$, 简记为 $C[a,b]$. 可以证明此空间是完备的, 这个空间是泛函中常见的例子.
\end{example}

\begin{definition}[压缩映射]
    称 $T: (\mathscr{X}, \rho) \to (\mathscr{X}, \rho)$ 是一个\textbf{压缩映射}, 如果存在 $0<\alpha<1$, 使得 $\rho(Tx,Ty) \leq \alpha \rho(x,y) \quad (\forall x,y \in \mathscr{X})$
\end{definition}

\begin{thm}[压缩映射原理]
    设 $(\mathscr{X}, \rho)$ 是一个完备的度量空间, $T$ 是 $(\mathscr{X}, \rho)$ 到其自身的一个压缩映射, 则 $T$ 在 $\mathscr{X}$ 上存在唯一的不动点.
\end{thm}
\begin{proof}
    任取初始点 $x_0 \in \mathscr{X}$, 考察序列 $x_{n+1} = Tx_{n}$, 下证其为基本列:
    \[|x_{n+p} - x_{n} | \leq \sum\limits_{i = 1}^p | x_i - x_{i+1}| \leq \sum\limits_{i=1}^p\alpha^{n+i-1} |x_1 - x_0| < \frac{\alpha^n}{1-\alpha} |x_1-x_0|\]
\end{proof}

利用压缩映射原理可以证明\textbf{常微分方程的一种初值问题的局部存在唯一性}与\textbf{隐函数存在定理}.

\subsection{完备化}
\begin{definition}[稠密子集]
    设 $(\mathscr{X},\rho)$ 是度量空间, 集合 $E \subset \mathscr{X}$ 叫作在 $\mathscr{X}$ 中的\textbf{稠密子集}, 如果 $\forall x \in \mathscr{X}, \forall \epsilon > 0, \exists z \in E$, 使得
     $\rho(x,z) < \epsilon$. 
\end{definition}

\begin{definition}[完备化空间]
    包含给定度量空间 $(\mathscr{X},\rho)$ 的最小的完备度量空间称为 $\mathscr{X}$ 的\textbf{完备化空间}, 其中最小的含义是: 任何一个以 $(\mathscr{X},\rho)$ 为子空间的完备度量空间都以此空间作为子空间.
\end{definition}

\begin{thm}
    如果 $(\mathscr{X}_1, \rho_1)$ 是一个以 $(\mathscr{X}, \rho)$ 为子空间的完备度量空间, $\rho_1 | _{\mathscr{X} \times \mathscr{X}} = \rho$, 并且 $\mathscr{X}$ 在 $\mathscr{X}_1$ 中稠密, 则 $\mathscr{X}_1$ 是 $\mathscr{X}$ 的完备化空间.
\end{thm}

\begin{thm}
    每一个度量空间都有一个完备化空间.
\end{thm}
\begin{proof}
    设 $(\mathscr{X}, \rho)$ 是一个度量空间, 考虑其上的基本列组成的等价类, 可以证明这是一个度量空间, 且是 $(\mathscr{X}, \rho)$ 的完备化空间.
\end{proof}

\subsection{列紧集}
\begin{definition}[列紧集]
    设 $(\mathscr{X}, \rho)$ 是一个度量空间, $A$ 是其一子集, 称 $A$ 是\textbf{列紧的}, 如果 $A$ 中的任意点列在 $\mathscr{X}$ 中有一个收敛子列. 
    若这个子列还收敛到 $A$ 中的点, 则称 $A$ 是\textbf{自列紧}的. 如果空间 $\mathscr{X}$ 是列紧的, 那么称 $\mathscr{X}$ 为\textbf{列紧空间}.
\end{definition}

\begin{definition}[ $\epsilon$ 网]
    设 $M$ 是 $(\mathscr{X}, \rho)$ 中的一个子集, $\epsilon > 0, N \subset M$. 如果对于 $\forall x \in M, \exists y \in N$ 使得 $\rho(x,y) < \epsilon$, 那么称 $N$ 是 $M$ 的一个\textbf{ $\epsilon$ 网}. 
    如果 $N$ 还是一个有穷集$($个数依赖于 $\epsilon$ $)$, 那么称 $N$ 为 $M$ 的一个\textbf{有穷 $\epsilon$ 网}.
\end{definition}

\begin{definition}[完全有界]
    集合 $M$ 称为是\textbf{完全有界}的, 如果 $\forall \epsilon > 0,$ 都存在 $M$ 的一个有穷 $\epsilon$ 网.
\end{definition}

\begin{thm}[Hausdorff]
    $($完备$)$度量空间中的集合 $M$ 是列紧的, 必须$($且仅需$)$ $M$ 是完全有界集.
\end{thm}

\begin{definition}[可分空间]
    一个度量空间若有可数的稠密子集, 那么称此度量空间是\textbf{可分的}.
\end{definition}

\begin{thm}
    完全有界的度量空间都是可分的.
\end{thm}
\begin{proof}
    取 $N_n$ 为 有穷的 $\frac{1}{n}$ 网, 则 $\bigcup\limits_{n=1}^{\infty} N_n$ 为一个可数的稠密子集.
\end{proof}

\begin{definition}[紧集]
    在拓扑空间 $\mathscr{X}$ 中, 集合 $M$ 称为是\textbf{紧的}, 如果 $\mathscr{X}$ 中每个覆盖 $M$ 的开集族中有有限个开集覆盖集合 $M$.
\end{definition}

\begin{thm}
    设 $(\mathscr{X}, \rho)$ 是一个度量空间, $M \subset \mathscr{X}$ 是紧的当且仅当 $M$ 是自列紧的.
\end{thm}

\subsection{赋范线性空间}
\begin{definition}[次线性泛函]
    设 $P:\mathscr{X} \to \mathbb{R}$ 是线性空间$\mathscr{X}$上的一个函数,若它满足
    \begin{enumerate}[label=(\roman*),font=\upshape]
        \item $P(x+y) \leq P(x) + P(y) \quad (\forall x, y \in \mathscr{X})$
        \item $P(\lambda x) = \lambda P(x) \quad (\forall \lambda > 0, \forall x \in \mathscr{X})$
    \end{enumerate}
    则称$P$为$\mathscr{X}$上的一个\textbf{次线性泛函}.
\end{definition}

\subsection{凸集与不动点}
\begin{definition}[Minkowski泛函]
    设$\mathscr{X}$是线性空间,$C$是$\mathscr{X}$上含有$\theta$的凸子集,在$\mathscr{X}$上规定一个取值于$[0,\infty]$的函数
    \[P(x) = \inf\{\lambda > 0 |\frac{x}{\lambda} \in C\} \quad (\forall x \in \mathscr{X})\]
    与$C$对应,称函数$P$为$C$的\textbf{$\mathrm{Minkowski}$泛函}.
\end{definition}

\subsection{内积空间}

\begin{thm}
    如果$C$是$\mathrm{Hilbert}$空间$\mathscr{X}$中的一个闭凸子集,
    那么在$C$上存在唯一元素$x_0$取到最小范数.
\end{thm}
\begin{proof}
    存在性: 设 $d= \inf_{z \in C}\|z\|$,取 $x_n$, 使得 $d\leq \|x_n\|\leq d + \frac{1}{n}$ 利用
    \[\|x_m - x_n\|^2 = 2(\|x_m\|^2+\|x_n\|^2)-4\|\frac{x_m + x_n}{2}\|^2\]
    可以证明 $\{x_n\}$ 是柯西列.\\
    唯一性: 同上利用极化恒等式可证唯一性.
\end{proof}
\begin{corollary}[$\mathrm{Hilbert}$空间闭凸子集中最佳逼近元存在且唯一]
    若C是$\mathrm{Hilbert}$空间$\mathscr{X}$中的一个闭凸子集,则对
    $\forall y \in \mathscr{X},\exists !x_0 \in C$,使得\[\|y-x_0\| = \inf_{x\in C}\|x-y\|\]
\end{corollary}
\begin{proof}
    将$C$平移$-y$之后,利用上面的定理即可.
\end{proof}
\begin{thm}[最佳逼近元的刻画]
    设$C$是内积空间$\mathscr{X}$中的闭凸子集, $\forall y \in \mathscr{X}$为了$x_0$是$y$在$C$上的最佳逼近元,必须且仅须它适合
    \[\mathrm{Re}(y-x_0,x_0-x)\geq 0,\forall x \in C\]
\end{thm}
\begin{proof}
    $\forall x \in C$,考虑函数$\varphi_x(t) = \|y-tx-(1-t)x_0\|^2$,为了$x_0$是$y$在$C$上的最佳逼近元,必须且仅需
    \[\varphi_x(t)\geq\varphi_x(0) \quad \forall x \in C ,\forall t \in[0,1]\]
    考察此二次函数的取值即可,这是容易地.
\end{proof}
\begin{corollary}
    设$M$是$\mathrm{Hilbert}$空间$\mathscr{X}$的一个闭线性子流形,$\forall x \in \mathscr{X}$,为了$y$是$x$在$M$上的最佳逼近元,必须且仅须它适合
    \[x-y \perp M-\{y\} \]
\end{corollary}
这个推论对线性子流形(子空间)中的最佳逼近元作了很到位的刻画,可以利用这个刻画得到最小二乘法的具体算法.
\begin{corollary}[正交分解]
    设$M$是$\mathrm{Hilbert}$空间$\mathscr{X}$中的一个闭线性子空间,那么$\forall x \in \mathscr{X}$,存在下列唯一的正交分解:
    \[x= y+z\quad (y\in M,z\in M^{\top})\]
\end{corollary}
\begin{proof}
    实际上,$y$为$x$在$M$上的最佳逼近元.
\end{proof}


\newpage

\section{线性算子与线性泛函}

\subsection{线性算子}
\begin{definition}[线性算子]
    设$\mathscr{X},\mathscr{Y}$是两个线性空间,$D$是$\mathscr{X}$的一个线性子空间.$T:D\to \mathscr{Y}$是一个映射,
    $D$称为定义域,也记作$D(T)$,$R(T)=\{Tx|x \in D\}$称为$T$的值域.如果
    \[T(\alpha x + \beta y) = \alpha Tx + \beta Ty\quad (\forall x,y \in D,\forall \alpha,\beta \in \mathbb{K})\]
    那么称$T$是一个\textbf{线性算子}.取值于实数$\mathbb{R}$(复数$\mathbb{C}$)的线性算子称为\textbf{实(复)线性泛函}.
\end{definition}
\begin{definition}[连续性]
    设$\mathscr{X},\mathscr{Y}$是$F^*$空间,$D(T)\subset \mathscr{X}$,称线性算子$T:D(T)\to \mathscr{Y}$在$x_0 \in D(T)$是\textbf{连续的},如果
    \[x_n \in D(T),\quad x_n \to x_0\Rightarrow Tx_n\to Tx_0\]
\end{definition}
\begin{definition}[有界性]
    设$\mathscr{X},\mathscr{Y}$都是$B^*$空间,称线性算子$T:\mathscr{X}\to \mathscr{Y}$是\textbf{有界的},如果存在常数$M\geq 0$,使得
    \[\|Tx\|_{\mathscr{Y}}\leq M\|x\|_{\mathscr{X}} \quad (\forall x \in \mathscr{X})\]
\end{definition}
\begin{thm}[有界等价于连续]
    设$\mathscr{X},\mathscr{Y}$都是$B^*$空间,为了线性算子$T$连续,必须且只须$T$有界.
\end{thm}
\begin{definition}[线性算子空间]
    用$\mathscr{L}(\mathscr{X},\mathscr{Y})$表示一切由$\mathscr{X}$到$\mathscr{Y}$的有界线性算子的全体,并规定
    \[\|T\| = \underset{x\in \mathscr{X}\setminus \theta}{\sup} \frac{\|Tx\|}{\|x\|} = \underset{\|x\|=1}{\sup}\|Tx\| \]
    为$T\in \mathscr{L}(\mathscr{X},\mathscr{Y})$的\textbf{范数},特别用$\mathscr{L}(\mathscr{X})$表示$\mathscr{L}(\mathscr{X},\mathscr{X})$,
    用$\mathscr{X}^*$表示$\mathscr{L}(\mathscr{X},\mathbb{K})$,即$\mathscr{X}^*$表示$\mathscr{X}$上的有界线性泛函全体.
\end{definition}
\begin{thm}[线性算子空间上的运算]
    设$\mathscr{X}$是$B^*$空间,$\mathscr{Y}$是$B$空间,若在$\mathscr{L}(\mathscr{X},\mathscr{Y})$上规定线性运算:
    \[(\alpha_1 T_1 + \alpha_2 T_2)(x) = \alpha_1 T_1 x + \alpha_2 T_2 x \quad (\forall x \in \mathscr{X}) \]
    其中$\alpha_1,\alpha_2\in \mathbb{K},T_1,T_2\in\mathscr{L}(\mathscr{X},\mathscr{Y})$,则$\mathscr{L}(\mathscr{X},\mathscr{Y})$按$\|T\|$
    构成一个$\mathrm{Banach}$空间.
\end{thm}
\begin{proof}
    主要是证范数的三角不等式与完备性,三角不等式基于算子范数的定义与$\mathscr{Y}$中范数的三角不等式.根据$\mathscr{Y}$的完备性可以定义极限算子.
\end{proof}

\subsection{Riesz 表示定理及其应用}
\begin{thm}[Riesz 表示定理 (Hilbert空间)]
    设 $f$ 是 $\mathrm{Hilbert}$ 空间 $\mathscr{X}$ 上的一个连续线性泛函,则必存在唯一的 $y_f \in \mathscr{X}$, 使得
    \[f(x) = (x,y_f)\quad (\forall x \in \mathscr{X})\]
\end{thm}
\begin{proof}
    考虑 $M\triangleq \{x\in \mathscr{X}|f(x) = 0\}$, 不妨设 $f$ 不恒等于 $0$, 由 $f$ 连续线性保证 $M$ 是一个真闭线性子空间.
    任取 $x_0 \perp M$, 正交分解定理保证了 $x_0$ 是存在的.由于 \[x-\frac{f(x)}{f(x_0)}x_0 \in M \quad (\forall x \in \mathscr{X})\]
    (由此可以看到 $M$ 是余 $1$ 维的) 于是 \[f(x)\frac{\|x_0\|^2}{f(x_0)} = (x,x_0)\]所以取 $y_f = (\overline{f(x_0)}/\|x_0\|^2) x_0$ 即可.
\end{proof}
实际上还有$\|f\| = \|y_f\|$.
\begin{thm}[共轭双线性函数的表示]
    设$\mathscr{X}$是一个$\mathrm{Hilbert}$空间,$a(x,y)$是$\mathscr{X}$上的共轭双线性函数,并$\exists M>0$.使得
    \[|a(x,y)| \leq M\|x\|\cdot \|y\|\quad (\forall x,y \in \mathscr{X})\]
    则存在唯一的$A \in \mathscr{L}(\mathscr{X})$,使得\[a(x,y) = (x,Ay) \quad (\forall x,y \in \mathscr{X})\]
    且\[\|A\| = \underset{(x,y) \in \mathscr{X}\times \mathscr{X}, x \neq \theta, y \neq \theta}{\sup}\frac{|a(x,y)|}{\|x\|\cdot \|y\|} \]
\end{thm}
Riesz表示定理有很多应用,这首先依赖于\rm{Hilbert}空间本身的良好性质,其次要求是连续线性泛函.

\paragraph{Laplace方程 $-\Delta u = f$ Dirichlet边值问题的弱解}\mbox{} 

设$\Omega\subset \mathbb{R}^n$是一个有界开区域,$f\in L^2(\Omega)$,称实函数$u$是
\begin{equation}\begin{cases}
    -\Delta u = f &\quad (\text{在}\Omega \text{内})\\u|_{\partial \Omega} = 0&
\end{cases}\tag{\#} \end{equation}
的一个\textbf{弱解}是指$u \in H_0^1(\Omega)$,满足
\[\int_{\Omega} \nabla u \cdot \nabla v \mathrm{d}x = \int_{\Omega}fv\mathrm{d}x \quad (\forall v \in H_0^1(\Omega))\]
这是因为:如果$u \in C^2(\overline{\Omega})$,并且是一个解,那么$\forall v \in C^2(\Omega),v|_{\partial \Omega} = 0$
\[\int_{\Omega}fv\mathrm{d}x = \int_{\Omega}-\Delta u \cdot v\mathrm{d}x 
\overset{\mathrm{Green}}{=} \int_{\Omega}\nabla u \cdot \nabla v \mathrm{d}x - \int_{\partial \Omega} \frac{\partial u}{\partial v} v \mathrm{d}\sigma
= \int_{\Omega}\nabla u \cdot \nabla v \mathrm{d}x\]
由于$\{v \in C^2(\Omega)|v|_{\Omega} = 0\}$在$H_0^1(\Omega)$中稠密,所以上式对$\forall v\in H_0^1(\Omega)$也成立,于是$u$是一个弱解.


对Laplace方程直接求解是很困难的,近代偏微分方程理论的基本方法是先求弱解,证其存在唯一,再证其光滑性,正因如此,泛函分析是研究近代偏微分方程理论必不可少的工具.
\begin{thm}[Laplace方程Dirichlet边值问题的弱解存在唯一]
    $\forall f \in L^2(\Omega)$,(\#)式所代表的$0$-Dirichlet问题(即边界条件为$0$)弱解存在唯一.
\end{thm}
\begin{proof}
    可以证明:(书上有证)\[(u,v) \triangleq \int_{\Omega} \nabla u \cdot \nabla v \mathrm{d}x\]
    是$H_0^1(\Omega)$上的一个内积,对$\forall v \in H_0^1(\Omega)$
    \[|\int_{\Omega}f\cdot v \mathrm{d}x|\leq (\int_{\Omega}|f|^2\mathrm{d}x)^{\frac{1}{2}} (\int_{\Omega}|v|^2\mathrm{d}x)^{\frac{1}{2}}
    \leq C \|f\|\cdot \|v\|_1\]
    其中$\|\cdot\|$与$\|\cdot\|_1$分别表示$L^2(\Omega)$与$H_0^1(\Omega)$上的范数.上式表明
    \[v \to \int_{\Omega}f\cdot v \mathrm{d}x\]
    是$H_0^1(\Omega)$上的一个连续线性泛函.应用Riesz表示定理,$\exists u \in H_0^1(\Omega)$,使得
    \[\int_{\Omega}fv\mathrm{d}x = (u_0,v)_1 = \int_{\Omega}\nabla u \cdot \nabla v\mathrm{d}x\quad (\forall v \in H_0^1(\Omega))\]
    从而$u_0$是一个弱解.唯一性略.
\end{proof}

\subsection{纲与开映射定理}
\begin{definition}[疏集,第一纲集,第二纲集]
    设$(\mathscr{X},\rho )$是一个度量空间,集合$E \subset \mathscr{X}$,称$E$是\textbf{疏}的,如果$\overline{E}$的内点是空的.
    集合$E$称为是\textbf{第一纲的},如果$E = \bigcup_{n=1}^\infty E_n$,其中$E_n$是疏集.不是第一纲的集合称为\textbf{第二纲集}.
\end{definition}
\begin{thm}[Baire]
    完备度量空间$(\mathscr{X}, \rho)$是第二纲集.
\end{thm}
\begin{proof}
    根据疏集去构造基本列,并使得基本列的极限不在$\mathscr{X}$中.
\end{proof}
\begin{thm}
    在$C[0,1]$中处处不可微的函数集合$E$的余集是第一纲集,特别地,$E$非空.
\end{thm}
\begin{thm}[开映射定理]
    设$\mathscr{X},\mathscr{Y}$都是$B$空间,若$T \in \mathscr{L}(\mathscr{X},\mathscr{Y})$是一个满射,则$T$是开映射.
\end{thm}
\begin{thm}[Banach]
    设$\mathscr{X},\mathscr{Y}$是$B$空间,若$T \in \mathscr{L}(\mathscr{X}, \mathscr{Y})$,它既是单射又是满射,
    那么$T^{-1} \in \mathscr{L}(\mathscr{X}, \mathscr{Y})$.
\end{thm}
\begin{definition}[闭线性算子]
    设$T$是$\mathscr{X} \to \mathscr{Y}$的线性算子,$D(T)$是其定义域.称$T$是\textbf{闭的}.是指由 $x_n \to x$ 以及 $Tx_n \to y$ 就能推出 $x \in D(T)$, 而且 $y = Tx$.
\end{definition}
\begin{corollary}[等价范数定理]
    设线性空间 $\mathscr{X}$ 上有两个范数 $\|\cdot\|_1$ 与 $\|\cdot\|_2$. 如果 $\mathscr{X}$ 关于这两个范数都构成 $B$ 空间, 
    而且 $\|\cdot\|_2$ 比 $\|\cdot\|_1$ 强, 则 $\|\cdot\|_2$ 与 $\|\cdot\|_1$ 必等价.
\end{corollary}
\begin{thm}[闭图像定理]
    设 $\mathscr{X}, \mathscr{Y}$ 是 $B$ 空间, 若 $T$ 是 $\mathscr{X} \to \mathscr{Y}$ 的闭线性算子,并且 $D(T)$ 是闭的, 则 $T$ 是连续的.
\end{thm}
\begin{thm}[一致有界定理]
    设 $\mathscr{X}$ 是 $B$ 空间, $\mathscr{Y}$ 是 $B^*$ 空间, 如果 $W \subset \mathscr{L}(\mathscr{X}, \mathscr{Y})$, 使得
    \[\underset{A \in W}{\sup} \|Ax\| < \infty \quad (\forall x \in \mathscr{X})\]
    那么存在常数 $M$, 使得 $\|A\| \leq M \quad (\forall A \in W)$.
\end{thm}
本小节后面的两个应用还没怎么细看,这节东西太多了,一下子看完不是很消化...另外一个原因是已经等不及赶紧进入下一节去看凸集分离定理了!
在最优化方法的课上就早有耳闻,如今终于可以学到了!
\subsection{Hahn-Banach定理}
\begin{thm}[实 Hahn-Banach 定理]
    设 $\mathscr{X}$ 是实线性空间, $p$ 是定义在 $\mathscr{X}$ 上的次线性泛函, $\mathscr{X}_0$ 是 $\mathscr{X}$ 的实线性子空间, 
    $f_0$ 是 $\mathscr{X}_0$ 上的实线性泛函并满足 $f(x_0) \leq p(x) \quad (\forall x \in \mathscr{X}_0)$. 
    那么 $\mathscr{X}$ 上必有一个实线性泛函 $f$, 满足:
    \begin{enumerate}[label=(\roman*),font=\upshape]
        \item $f(x) \leq p(x) \quad (\forall x \in \mathscr{X})$
        \item $f(x) = f_0(x) \quad (\forall x \in \mathscr{X}_0)$
    \end{enumerate}
\end{thm}
\begin{proof}
    每次延拓一维, 这等价于在一个已有定义的子空间之外的一个向量上定义取值, 但是取值需要满足被给定的次线性泛函 $p(x)$ 控制, 
    最后用Zorn引理可以保证最后能延拓到整个空间上去. 
\end{proof}
\begin{thm}[复 Hahn-Banach 定理]
    设 $\mathscr{X}$ 是复线性空间, $p$ 是 $\mathscr{X}$ 上的半范数. $\mathscr{X}_0$ 是 $\mathscr{X}$ 的线性子空间, 
    $f$ 是 $\mathscr{X}_0$ 上的线性泛函, 并满足 $|f_0(x)| \leq p(x)$, $\forall x \in \mathscr{X}_0$, 那么 $\mathscr{X}$ 上
    必有一个线性泛函 $f$ 满足:
    \begin{enumerate}[label=(\roman*),font=\upshape]
        \item $|f(x)| \leq p(x) \quad (\forall x \in \mathscr{X})$
        \item $f(x) = f_0(x) \quad (\forall x \in \mathscr{X}_0)$
    \end{enumerate}
\end{thm}
\begin{thm}[Hahn-Banach 定理]
    设 $\mathscr{X}$ 是 $B^*$ 空间, $\mathscr{X}_0$ 是 $\mathscr{X}$ 的线性子空间, 
    $f_0$ 是定义在 $\mathscr{X}_0$ 上的有界线性泛函, 则在 $\mathscr{X}$ 上必有有界线性泛函 $f$ 满足:
    \begin{enumerate}[label=(\roman*),font=\upshape]
        \item $f(x) = f_0(x) \quad (\forall x \in \mathscr{X})$
        \item $\|f\| = \|f_0\|_0$
    \end{enumerate}
    其中 $\|f_0\|_0$ 表示 $f_0$ 在 $\mathscr{X}_0$ 上的范数.
\end{thm}
\begin{proof}
    在 $\mathscr{X}$ 上定义 $p(x) \triangleq \|f_0\|_0 \cdot \|x\|$, 那么 $p(x)$ 是 $\mathscr{X}$ 上的半范数, 
    从而根据 实 Hanh-Banach 定理, 必存在 $\mathscr{X}$ 上的线性泛函 $f(x)$, 满足
    \[f(x) = f_0(x) \quad (\forall x \in \mathscr{X}_0)\]并且
    \[ |f(x)| \leq p(x) = \|f_0\|_0 \cdot \|x\| \quad (\forall x \in \mathscr{X})\]
    因此 $\|f\| = \|f_0\|_0$.
\end{proof}
\begin{thm}
    设 $\mathscr{X}$ 是 $B^*$ 空间, $M$ 是 $\mathscr{X}$ 的线性子空间, 若 $x_0 \in \mathscr{X}$, 且
    \[d \triangleq \rho (x_0, M) > 0\] 则必 $\exists f \in \mathscr{X}^*$ 适合条件:
    \begin{enumerate}[label=(\roman*),font=\upshape]
        \item $f(x) = 0 \quad (\forall x \in M)$
        \item $f(x_0) = d$
        \item $\|f\| =1$
    \end{enumerate}
\end{thm}
\subsection{共轭空间、弱收敛、自反空间}
\subsection{线性算子的谱}

\end{document}