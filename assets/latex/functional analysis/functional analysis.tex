\documentclass[12pt,a4paper]{article}

% 中文支持
\usepackage[UTF8]{ctex}

% 数学宏包
\usepackage{amsmath,amsfonts,amssymb,blkarray,mathrsfs,amsthm}

%插入代码
\usepackage{listings}
\usepackage[most]{tcolorbox}

% 图片和表格
\usepackage{graphicx}
\usepackage{float}
\usepackage{booktabs}
\usepackage{subcaption}

% 页面布局
\usepackage{geometry}
\geometry{left=2.5cm,right=2.5cm,top=3cm,bottom=3cm}
\usepackage{multirow}


% 超链接
\usepackage{hyperref}
\hypersetup{
    colorlinks=true,
    linkcolor=black,
    citecolor=black,
    urlcolor=red
}


% 页眉和页脚
\usepackage{fancyhdr}
\pagestyle{fancy}
\fancyhf{}
\rhead{泛函分析笔记}
\lhead{WRL}
\rfoot{第 \thepage 页}

% 颜色
\usepackage{xcolor}


% 定义定理环境
\newtheorem{thm}{定理}[subsection]  % 在每个章节重新编号
\newtheorem{lemma}{引理}[subsection]    % 在每个章节重新编号
\newtheorem{corollary}{推论}[subsection] % 在每个章节重新编号
\newtheorem{def}{定义}[subsection] % 在每个章节内自动编号



\begin{document}
\begin{center}
\section*{摘要}
\end{center}

本来寒假打算在个人主页上面更新泛函分析的自学笔记,但在markdown文件中插入latex真不太方便,
故单独用\LaTeX文件来记泛函笔记。

我自学使用的教材为张恭庆老师的《泛函分析讲义(上)》,在此笔记中主要记录书中的核心内容,配以心得体会。


{\centering\tableofcontents}

\newpage
\section{度量空间}
\subsection{压缩映射原理}
\subsection{完备化}
\subsection{列紧集}
\subsection{赋范线性空间}
\subsection{凸集与不动点}
\subsection{内积空间}

\begin{thm}
    如果$C$是\rm{Hilbert}空间$\mathscr{X}$中的一个闭凸子集,
    那么在$C$上存在唯一元素$x_0$取到最小范数.
\end{thm}
\begin{proof}
    存在性: 设$d= \inf_{z \in C}\|z\|$,取$x_n$,使得$d\leq \|x_n\|\leq d + \frac{1}{n}$利用
    \[\|x_m - x_n\|^2 = 2(\|x_m\|^2+\|x_n\|^2)-4\|\frac{x_m + x_n}{2}\|^2\]
    可以证明$\{x_n\}$是柯西列.\\
    唯一性: 同上利用极化恒等式可证唯一性.
\end{proof}
\begin{corollary}
    若C是\rm{Hilbert}空间$\mathscr{X}$中的一个闭凸子集,则对
    $\forall y \in \mathscr{X},\exists !x_0 \in C$,使得\\$\|y-x_0\| = \inf_{x\in C}\|x-y\|$
\end{corollary}
\begin{proof}
    将$C$平移$-y$之后,利用上面的定理即可.
\end{proof}

\newpage
\section{线性算子与线性泛函}
\subsection{线性算子的概念}
\subsection{Riesz表示定理及其应用}
\subsection{纲与开映射定理}
\subsection{Hahn-Banach定理}
\subsection{共轭空间、弱收敛、自反空间}
\subsection{线性算子的谱}

\end{document}