\documentclass[12pt,a4paper]{article}

% 中文支持
\usepackage[UTF8]{ctex}

% 数学宏包
\usepackage{amsmath,amsfonts,amssymb,blkarray,mathrsfs,amsthm}

%插入代码
\usepackage{listings}
\usepackage[most]{tcolorbox}

% 图片和表格
\usepackage{graphicx}
\usepackage{float}
\usepackage{booktabs}
\usepackage{subcaption}

% 页面布局
\usepackage{geometry}
\geometry{left=2.5cm,right=2.5cm,top=3cm,bottom=3cm}
\usepackage{multirow}


% 超链接
\usepackage{hyperref}
\hypersetup{
    colorlinks=true,
    linkcolor=black,
    citecolor=black,
    urlcolor=red
}


% 页眉和页脚
\usepackage{fancyhdr}
\pagestyle{fancy}
\fancyhf{}
\rhead{概率论笔记}
\lhead{WRL}
\rfoot{第 \thepage 页}

% 颜色
\usepackage{xcolor}

\newtheorem{thm}{定理}[subsection]  % 在每个章节重新编号
\newtheorem{lemma}{引理}[subsection]    % 在每个章节重新编号
\newtheorem{corollary}{推论}[subsection] % 在每个章节重新编号
\newtheorem{definition}{定义}[subsection] % 在每个章节内自动编号

\begin{document}
\begin{center}
\section*{摘要}
\end{center}

我自学使用的教材为《概率论基础》,在此笔记中主要记录书中的核心内容,配以心得体会。


{\centering\tableofcontents}

\newpage
\section{事件与概率}
\subsection{样本空间与事件}

\begin{definition}
    对于(随机)试验,可能出现的结果称为\textbf{样本点},样本点全体构成\textbf{样本空间}.
\end{definition}
在概率论中一般假定样本空间是给定的,这是必要的抽象,是我们能更好地把我住随机变量的本质,类比于线性空间.
\begin{definition}
    \textbf{事件}定义为样本点的某个集合,称事件发生当且仅当它所包含的某一个样本点出现.
\end{definition}
在此定义下,集合的包含关系诱导了事件的包含关系,补集对应于逆事件,或称对立事件,两个集合的交意味着两个事件的交意味着两个事件同时发生,并集意味着至少发生一个.
\begin{definition}
    
\end{definition}
\newpage
\section{条件概率与统计独立性}
\subsection{}
\subsection{}


\newpage
\section{随机变量与分布函数}
\subsection{}
\subsection{}



\end{document}