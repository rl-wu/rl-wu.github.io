\documentclass[12pt,a4paper]{article}

% 中文支持
\usepackage[UTF8]{ctex}

% 数学宏包
\usepackage{amsmath,amsfonts,amssymb,blkarray}

%插入代码
\usepackage{listings}
\usepackage[most]{tcolorbox}

% 图片和表格
\usepackage{graphicx}
\usepackage{float}
\usepackage{booktabs}
\usepackage{subcaption}

% 页面布局
\usepackage{geometry}
\geometry{left=2.5cm,right=2.5cm,top=3cm,bottom=3cm}
\usepackage{multirow}


% 超链接
\usepackage{hyperref}
\hypersetup{
    colorlinks=true,
    linkcolor=black,
    citecolor=black,
    urlcolor=red
}


% 页眉和页脚
\usepackage{fancyhdr}
\pagestyle{fancy}
\fancyhf{}
\rhead{泛函分析笔记}
\lhead{WRL}
\rfoot{第 \thepage 页}

% 颜色
\usepackage{xcolor}



\begin{document}
\begin{center}
\section*{摘要}
\end{center}

我自学使用的教材为柳彬老师的《常微分方程》,在此笔记中主要记录书中的核心内容,配以心得体会。


{\centering\tableofcontents}

\newpage
\section{微分方程的基本概念}
\subsection{微分方程的定义}
\subsection{几何解释}

\newpage
\section{初等积分法}
\subsection{恰当方程}
\subsection{变量分离方程}
\subsection{一阶线性微分方程}
\subsection{积分因子}
\subsection{一阶隐式微分方程}
\subsection{应用举例}

\newpage
\section{解的存在唯一性}
\subsection{准备知识}
\subsection{Picard定理}
\subsection{Peano定理}
\subsection{解的延伸}
\subsection{比较定理}
\subsection{奇解}
\subsection{包络}


\end{document}