\documentclass[12pt,a4paper]{article}

% 中文支持
\usepackage[UTF8]{ctex}

% 数学宏包
\usepackage{amsmath,amsfonts,amssymb,blkarray}

%插入代码
\usepackage{listings}
\usepackage[most]{tcolorbox}

% 图片和表格
\usepackage{graphicx}
\usepackage{float}
\usepackage{booktabs}
\usepackage{subcaption}

% 页面布局
\usepackage{geometry}
\geometry{left=2.5cm,right=2.5cm,top=3cm,bottom=3cm}
\usepackage{multirow}


% 超链接
\usepackage{hyperref}
\hypersetup{
    colorlinks=true,
    linkcolor=black,
    citecolor=black,
    urlcolor=red
}


% 页眉和页脚
\usepackage{fancyhdr}
\pagestyle{fancy}
\fancyhf{}
\rhead{泛函分析笔记}
\lhead{吴睿林}
\rfoot{第 \thepage 页}

% 颜色
\usepackage{xcolor}



\begin{document}
\begin{center}
\section*{摘要}
\end{center}

本来寒假打算在个人主页上面更新泛函分析的学习笔记,但在markdown文件中插入latex真不太方便,
故单独用\LaTeX文件来记泛函笔记。

本文件

{\centering\tableofcontents}

\newpage
\section{度量空间}
\subsection{压缩映射原理}
\subsection{完备化}
\subsection{列紧集}
\subsection{赋范线性空间}
\subsection{凸集与不动点}
\subsection{内积空间}


\end{document}