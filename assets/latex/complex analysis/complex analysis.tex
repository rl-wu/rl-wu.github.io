\documentclass[12pt,a4paper]{article}

% 中文支持
\usepackage[UTF8]{ctex}

% 数学宏包
\usepackage{amsmath,amsfonts,amssymb,blkarray}

%插入代码
\usepackage{listings}
\usepackage[most]{tcolorbox}

% 图片和表格
\usepackage{graphicx}
\usepackage{float}
\usepackage{booktabs}
\usepackage{subcaption}

% 页面布局
\usepackage{geometry}
\geometry{left=2.5cm,right=2.5cm,top=3cm,bottom=3cm}
\usepackage{multirow}


% 超链接
\usepackage{hyperref}
\hypersetup{
    colorlinks=true,
    linkcolor=black,
    citecolor=black,
    urlcolor=red
}


% 页眉和页脚
\usepackage{fancyhdr}
\pagestyle{fancy}
\fancyhf{}
\rhead{复变函数笔记}
\lhead{WRL}
\rfoot{第 \thepage 页}

% 颜色
\usepackage{xcolor}


\newtheorem{thm}{定理}[section]  % 在每个章节重新编号
\newtheorem{lemma}{引理}[section]    % 在每个章节重新编号
\newtheorem{corollary}{推论}[section] % 在每个章节重新编号
\newtheorem{definition}{定义}[section] % 在每个章节内自动编号
\newtheorem{example}{例}[section] % 在每个章节内自动编号



\begin{document}
\begin{center}
\section*{摘要}
\end{center}


我自学使用的教材为谭小江, 伍胜健的《复变函数简明教程》,在此笔记中主要记录书中的核心内容,配以心得体会。


{\centering\tableofcontents}

\newpage
\section{Cauchy 定理和 Cauchy 公式}
\begin{thm}[Cauchy 定理]
    设 $\Omega$ 是 $\mathbb{C}$ 中以有限条逐段光滑曲线为边界的有界区域, 函数 $f(z)$ 在 $\overline{\Omega}$ 上连续, 在 $\Omega$ 内解析, 则
    \[\int_{\partial \Omega} f(z) \mathrm{d}z = 0\]
\end{thm}

\begin{thm}[Cauchy 公式]
    设 $\Omega$ 是由有限条逐段光滑曲线为边界的有界区域, $f(z)$ 在 $\overline{\Omega}$ 上连续, 在 $\Omega$ 内解析, 则 $\forall z \in \Omega$, 
    \[f(z) = \frac{1}{2\pi \mathrm{i}} \int_{\partial \Omega} \frac{f(\omega)}{\omega - z} \mathrm{d} \omega\]
\end{thm}

\begin{thm}
    函数 $f(z)$ 在区域 $\Omega$ 上解析的充分必要条件是 $\forall z_0 \in \Omega$, $f(z)$ 可在 $z_0$ 的邻域上展开为 $(z - z_0)$ 的幂级数. 其中在 $z_0$ 的邻域幂级数展开的形式为:
    \[f(z) = \sum\limits_{n = 0}^{+\infty} \left[ \frac{1}{2\pi \mathrm{i}} \int_{|\omega - z_0| = r} \frac{f(\omega)}{(\omega - z_0)^{n+1}} \mathrm{d} \omega \right] (z-z_0)^n\]
\end{thm}   

\begin{thm}[Morera 定理]
    设 $\Omega \subset \mathbb{C}$ 为一个区域, $f(z)$ 在 $\Omega$ 内连续. 则 $f(z)$ 在 $\Omega$ 内解析的充分必要条件是对 $\Omega$ 中任意由逐段光滑曲线为边界围成的有界区域 $D$, 如果 $\overline{D} \subset \Omega$, 则
    \[\int_{\partial D} f(\omega) \mathrm{d} \omega = 0\]
\end{thm}

\section{利用幂级数研究解析函数}

\begin{thm}
    设 $f(z)$ 在区域 $\Omega$ 内解析, 如果存在 $z_0 \in \Omega$, 使得
    \[f(z_0) = f^{\prime}(z_0) = f^{\prime \prime} (z_0) = \cdots = f^{n}(z_0) = \cdots = 0\]
    则 $f(z)$ 在 $\Omega$ 上恒为零.
\end{thm}

\begin{corollary}
    设 $f(z)$ 是区域 $\Omega$ 上不为常数的解析函数, 则 $\forall z_0 \in \Omega$, 存在正整数 $m$, 使得
    \[f^{\prime}(z_0) = f^{\prime \prime}(z_0) = \cdots = f^{m-1}(z_0) = 0, \text{而}\  f^{m}(z_0) \neq 0\]
    这时存在 $z_0$ 的邻域 $O$, 使得 $f(z)$ 在 $O$ 上可表示为
    \[f(z) - f(z_0) = (z - z_0)^{m} g(z)\] 其中 $g(z)$ 在 $O$ 上解析, 且 $g(z_0) \neq 0$.
\end{corollary}

\begin{thm}[开映射定理]
    如果 $f(z)$ 是区域 $\Omega$ 山不为常数的解析函数, 则 $f(z)$ 将 $\Omega$ 中的开集映为开集.
\end{thm}

\begin{thm}
    如果 $f(z)$ 是区域 $\Omega$ 上的单叶解析函数, 则 $f(\Omega)$ 是 $\mathbb{C}$ 中的开集, 因而是区域;
    $f^{\prime}(z)$ 在 $\Omega$ 上处处不为零. 因此 $f^{-1}: f(\Omega) \to \Omega$ 是解析的; $f: \Omega \to f(\Omega)$ 是解析同胚.
\end{thm}

\begin{thm}[最大模原理]
    如果 $f(z)$ 是区域 $\Omega$ 上不为常数的解析函数, 则 $|f(z)|$ 在 $\Omega$ 内无极大值点.
\end{thm}

\begin{thm}[代数学基本定理]
    设 \[P(z) = a_n z^n + a_{n-1} z^{n-1} + \cdots + a_0\]
    是一 $n$ 次多项式, 其中 $n \geq 1$, $a_n \neq 0$, 则方程 $P(z) = 0$ 在 $\mathbb{C}$ 中有解.
\end{thm}


\end{document}