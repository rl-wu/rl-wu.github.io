\documentclass[12pt,a4paper]{article}

% 中文支持
\usepackage[UTF8]{ctex}

% 数学宏包
\usepackage{amsmath,amsfonts,amssymb,blkarray,mathrsfs,amsthm,bm}

%插入代码
\usepackage{listings}
\usepackage[most]{tcolorbox}

% 图片和表格
\usepackage{graphicx}
\usepackage{float}
\usepackage{booktabs}
\usepackage{subcaption}

% 页面布局
\usepackage{geometry}
\geometry{left=2.5cm,right=2.5cm,top=3cm,bottom=3cm}
\usepackage{multirow}


% 超链接
\usepackage{hyperref}
\hypersetup{
    colorlinks=true,
    linkcolor=black,
    citecolor=black,
    urlcolor=red
}


% 页眉和页脚
\usepackage{fancyhdr}
\pagestyle{fancy}
\fancyhf{}
\rhead{复变函数笔记}
\lhead{WRL}
\rfoot{第 \thepage 页}

% 颜色
\usepackage{xcolor}


\newtheorem{thm}{定理}[section]  % 在每个章节重新编号
\newtheorem{lemma}{引理}[section]    % 在每个章节重新编号
\newtheorem{corollary}{推论}[section] % 在每个章节重新编号
\newtheorem{definition}{定义}[section] % 在每个章节内自动编号
\newtheorem{example}{例}[section] % 在每个章节内自动编号



\begin{document}
\begin{center}
\section*{摘要}
\end{center}


我自学使用的教材为谭小江, 伍胜健的《复变函数简明教程》,在此笔记中主要记录书中的核心内容,配以心得体会。


{\centering\tableofcontents}

\newpage
\section{Cauchy 定理和 Cauchy 公式}
\begin{thm}[Cauchy 定理]
    设 $\Omega$ 是 $\mathbb{C}$ 中以有限条逐段光滑曲线为边界的有界区域, 函数 $f(z)$ 在 $\overline{\Omega}$ 上连续, 在 $\Omega$ 内解析, 则
    \[\int_{\partial \Omega} f(z) \mathrm{d}z = 0\]
\end{thm}

\begin{thm}[Cauchy 公式]
    设 $\Omega$ 是由有限条逐段光滑曲线为边界的有界区域, $f(z)$ 在 $\overline{\Omega}$ 上连续, 在 $\Omega$ 内解析, 则 $\forall z \in \Omega$, 
    \[f(z) = \frac{1}{2\pi \mathrm{i}} \int_{\partial \Omega} \frac{f(\omega)}{\omega - z} \mathrm{d} \omega\]
\end{thm}

\begin{thm}
    函数 $f(z)$ 在区域 $\Omega$ 上解析的充分必要条件是 $\forall z_0 \in \Omega$, $f(z)$ 可在 $z_0$ 的邻域上展开为 $(z - z_0)$ 的幂级数. 其中在 $z_0$ 的邻域幂级数展开的形式为:
    \[f(z) = \sum\limits_{n = 0}^{+\infty} \left[ \frac{1}{2\pi \mathrm{i}} \int_{|\omega - z_0| = r} \frac{f(\omega)}{(\omega - z_0)^{n+1}} \mathrm{d} \omega \right] (z-z_0)^n\]
\end{thm}   

\begin{thm}[Morera 定理]
    设 $\Omega \subset \mathbb{C}$ 为一个区域, $f(z)$ 在 $\Omega$ 内连续. 则 $f(z)$ 在 $\Omega$ 内解析的充分必要条件是对 $\Omega$ 中任意由逐段光滑曲线为边界围成的有界区域 $D$, 如果 $\overline{D} \subset \Omega$, 则
    \[\int_{\partial D} f(\omega) \mathrm{d} \omega = 0\]
\end{thm}

\section{利用幂级数研究解析函数}

\begin{thm}
    设 $f(z)$ 在区域 $\Omega$ 内解析, 如果存在 $z_0 \in \Omega$, 使得
    \[f(z_0) = f^{\prime}(z_0) = f^{\prime \prime} (z_0) = \cdots = f^{n}(z_0) = \cdots = 0\]
    则 $f(z)$ 在 $\Omega$ 上恒为零.
\end{thm}

\begin{corollary}
    设 $f(z)$ 是区域 $\Omega$ 上不为常数的解析函数, 则 $\forall z_0 \in \Omega$, 存在正整数 $m$, 使得
    \[f^{\prime}(z_0) = f^{\prime \prime}(z_0) = \cdots = f^{m-1}(z_0) = 0, \text{而}\  f^{m}(z_0) \neq 0\]
    这时存在 $z_0$ 的邻域 $O$, 使得 $f(z)$ 在 $O$ 上可表示为
    \[f(z) - f(z_0) = (z - z_0)^{m} g(z)\] 其中 $g(z)$ 在 $O$ 上解析, 且 $g(z_0) \neq 0$.
\end{corollary}

\begin{thm}[开映射定理]
    如果 $f(z)$ 是区域 $\Omega$ 山不为常数的解析函数, 则 $f(z)$ 将 $\Omega$ 中的开集映为开集.
\end{thm}

\begin{thm}
    如果 $f(z)$ 是区域 $\Omega$ 上的单叶解析函数, 则 $f(\Omega)$ 是 $\mathbb{C}$ 中的开集, 因而是区域;
    $f^{\prime}(z)$ 在 $\Omega$ 上处处不为零. 因此 $f^{-1}: f(\Omega) \to \Omega$ 是解析的; $f: \Omega \to f(\Omega)$ 是解析同胚.
\end{thm}

\begin{thm}[最大模原理]
    如果 $f(z)$ 是区域 $\Omega$ 上不为常数的解析函数, 则 $|f(z)|$ 在 $\Omega$ 内无极大值点.
\end{thm}

\begin{thm}[代数学基本定理]
    设 \[P(z) = a_n z^n + a_{n-1} z^{n-1} + \cdots + a_0\]
    是一 $n$ 次多项式, 其中 $n \geq 1$, $a_n \neq 0$, 则方程 $P(z) = 0$ 在 $\mathbb{C}$ 中有解.
\end{thm}

\begin{proof}
    容易推出 \[ \lim_{z \to \infty} |P(z)| = + \infty\]
    取 $R$ 充分大, 使得 \[ \min\{|P(z)| | |z| = R\} > |P(0)| \]
    设 $z_0$ 是 $|P(z)|$ 在闭圆盘 $\overline{D(0,R)}$ 内的最小值点, 由假设知 $z_0 \in D(0,R)$. 因为 $P(z)$ 解析且不为常数, 所以 $P(D(0,R))$ 是 $\mathbb{C}$ 中的开集. 
    从而 $P(z_0)$ 是 $P(D(0,R))$ 的内点, 如果 $P(z_0) \neq 0$, 则 $|P(z_0)|$ 不能是最小, 此矛盾说明 $P(z_0) = 0$.

\end{proof}

\section{Cauchy不等式}

\begin{thm}[Cauchy 不等式]
    设 $f(z)$ 在区域 $\Omega$ 上解析, 且在 $\Omega$ 上 $|f(z)| \leq M$, 则 $\forall z_0 \in \Omega, 0 < r \leq \mathrm{dist}(z_0, \partial \Omega)$, 恒有
    \[ |f^{(n)}(z_0)| \leq \frac{n!M}{r^n}\]
\end{thm}

\section{Schwarz 引理}

\begin{thm}[Schwarz 引理]
    设 $f(z)$ 是单位圆盘 $D(0,1)$ 到自身的解析映射, 满足 $f(0) = 0 $, 则
    \begin{enumerate}
        \item $\forall z \in D(0,1), |f(z)| \leq |z|, |f^{\prime}(0)| \leq 1$
        \item 存在 $z_0 \neq 0$, 使得 $|f(z_0)| = |z_0|$ 或 $f^{\prime}(0)| = 1$ 的充分必要条件是 $f(z) = \mathrm{e}^{\mathrm{i} \theta} z$
    \end{enumerate}
\end{thm}

\begin{proof}
    设 $f(z) = \sum\limits_{i \geq 0} a_i z^i$ 是 $f(z)$ 在 $z=0$ 处展开的幂级数, 由 $f(0) = 0$ 得到 $a_0 = 0$, 因此
    \[\frac{f(z)}{z} = a_1 + a_2 z + \cdots + a_n z^{n-1} + \cdots\]
    在单位圆盘 $D(0,1)$ 内解析, 当 $z_0 \in D(0,1)$ 固定后, 任取 $r$ 使得 $|z_0| < r < 1$, 则由\textbf{最大模原理}知 $|\frac{f(z)}{z}|$ 在圆盘 $D(0,r)$ 的边界上取到最大值, 因此
    \[ \mid \frac{f(z)}{z} \mid \leq \frac{1}{r}\]
    令 $r \to 1$ 得到 $|f(z_0)| \leq |z_0|$, 特别地
    \[|f^{\prime}(0)| = |\frac{f(z)}{z}|_{z=0} | \leq 1\]
    如果存在 $z_0 \neq 0$ 使得 $|f(z_0)| = |z_0|$ 或 $|f(0)^{\prime}| = 1$, 则 $|\frac{f(z)}{z}|$ 在 $z_0$ 或 $0$ 处取到最大值 $1$, 由最大模原理知 $f(z) = \mathrm{e}^{\mathrm{i}\theta} z$
\end{proof}

\section{Laurent 级数}

\begin{thm}
 函数 $f(z)$ 在圆环区域 $D(z_0, r, R)$ 内解析的充分必要条件是 $f(z)$ 可在 $D(z_0, r, R)$ 上展开为关于 $z-z_0$ 的 $\mathrm{Laurent}$ 级数.
\end{thm}

\begin{definition}
    如果函数 $f(z)$ 在 $z_0$ 的空心邻域上解析, 即存在 $\epsilon > 0$, 使 $f(z)$ 在 $D_0(z_0, \epsilon)$ 上解析, 则 $z_0$ 称为 $f(z)$ 的\textbf{孤立奇点}. 
    如果存在 $R_0>0$, 使得 $f(z)$ 在 $\mathbb{C} - \overline{D(0, R_0)}$ 上解析, 则称 $\infty$ 是 $f(z)$ 的一个孤立奇点. 例如任何整函数都以 $\infty$ 为孤立奇点.
\end{definition}

\begin{definition}
    设 $z_0$ 是 $f(z)$ 的孤立零点.
    
    $(1)$ 如果存在 $c \in \mathbb{C}$ 使得函数 \[g(z) = \begin{cases}f(z) & z \neq z_0 \\ c & z = z_0 \end{cases}\] 
    在 $z_0$ 的邻域上解析, 则称 $f(z)$ 可\textbf{解析开拓}到 $z_0$ 处, 并称 $z_0$ 为 $f(z)$ 的\textbf{可去奇点}.

    $(2)$ 如果 $f(z)$ 不能解析开拓到 $z_0$ 处, 但 $\frac{1}{f(z)}$ 可解析开拓到 $z_0$, 则 $z_0$ 称为 $f(z)$ 的\textbf{极点}.

    $(3)$ 如果 $z_0$ 既不是 $f(z)$ 的可去奇点, 也不是 $f(z)$ 的极点, 则称 $z_0$ 为 $f(z)$ 的\textbf{本性奇点}.
\end{definition}

下面写点对于上述定义的刻画:

\begin{thm}
    设 $z_0$ 是 $f(z)$ 的孤立零点, 则下面的条件等价:

    $(1)$ $z_0$ 是 $f(z)$ 的可去奇点

    $(2)$ $\lim_{z \to z_0} f(z)$ 在 $\mathbb{C}$ 中存在

    $(3)$ $f(z)$ 在 $f(z)$ 的邻域上有界

    $(4)$ $f(z)$ 在 $z_0$ 的 $\mathrm{Larurent}$ 展式的主部为零
\end{thm}

\begin{thm}
    设 $z_0$ 是 $f(z)$ 的孤立奇点, 则下面条件等价:

    $(1)$ $z_0$ 是 $f(z)$ 的极点

    $(2)$ $z_0$ 是 $\frac{1}{f(z)}$ 的孤立零点

    $(3)$ $\lim_{z \to z_0} f(z) = \infty$

    $(4)$ $f(z)$ 在 $z_0$ 处 $\mathrm{Laurent}$ 展式的主部中有且仅有有限项不为零
\end{thm}

\begin{proof}
    $(1) \Rightarrow (2)$: 由于 $\frac{1}{f(z)}$ 可解析开拓到 $z_0$, 从而由前面一个定理可知 $\lim_{z \to z_0} \frac{1}{f(z)}$ 在 $\mathbb{C}$ 中存在, 但由于 $f(z)$ 不能解析开拓到 $z_0$, 因此必有
    \[ \lim_{z \to z_0} \frac{1}{f(z)} = 0\]

    $(2) \Rightarrow (3)$: 显然

    $(3) \Rightarrow (4)$: 由 $\lim_{z \to z_0} f(z) = \infty$, 我们得到 $z_0$ 是 $\frac{1}{f(z)}$ 的孤立零点. 设其是 $m$ 阶零点, 则 $\frac{1}{f(z)}$ 在 $z_0$ 邻域可展开为
    \[\frac{1}{f(z)} = (z-z_0)^m \sum\limits_{n = 0}^{+\infty} b_n (z-z_0)^n = (z-z_0)^m g(z)\]
    其中 $g(z)$ 在 $z_0$ 的邻域内解析且处处不为零. 因此 \[f(z) = \frac{1}{(z-z_0)^m} \cdot \frac{1}{g(z)}\]
    由于 $\frac{1}{g(z)}$ 在 $z_0$ 的邻域内解析, 我们有 \[\frac{1}{g(z)} = c_0 + c_1 (z-z_0) + \cdots\]
    其中 $c_0 \neq 0$, 从而推出 \[f(z) = \frac{c_0}{(z-z_0)^m} + \frac{c_1}{(z-z_0)^{m-1}} + \cdots\] 
    
    $(4) \Rightarrow (1)$: 设 $f(z)$ 在 $z_0$ 处 $\mathrm{Laurent}$ 展式的主部为 
    \[\frac{a_{-m}}{(z-z_0)^m} + \frac{a_{-m+1}}{(z-z_0)^{m-1}} + \cdots + \frac{a_{-1}}{(z-z_0)}\]
    其中 $a_{-m} \neq 0$, 则
    \[(z-z_0)^m f(z) \overset{\text{记为}}{=} g(z)\] 在 $z_0$ 处解析, 且 $g(z_0) \neq 0$, 因此 
    \[ f(z) = \frac{1}{(z-z_0)^m} \cdot g(z) \] 不能解析开拓到 $z_0$, 但
    \[\frac{1}{f(z)} = \frac{(z-z_0)^m}{g(z)}\] 可解析开拓到 $z_0$, 即 $z_0$ 是 $f(z)$ 的极点.
\end{proof}

\begin{thm}
    设 $z_0$ 是 $f(z)$ 的孤立奇点, 则下面的条件等价:

    $(1)$ $z_0$ 是 $f(z)$ 的本性奇点

    $(2)$ $\lim_{z \to z_0} f(z)$ 在 $\overline{\mathbb{C}} = \mathbb{C} \cup \{\infty\}$ 中不存在
    
    $(3)$ $f(z)$ 在 $z_0$ 的 $\mathrm{Laurent}$ 展式中主部有无穷多项不为零
\end{thm}

\begin{thm}[Weierstrass 定理]
    如果 $z_0$ 是 $f(z)$ 的本性奇点, 则 $\forall \epsilon > 0$, $f(D_0(z_0, \epsilon))$ 都是 $\mathbb{C}$ 中的稠密子集.
\end{thm}

\begin{proof}
    若结论不成立, 则存在 $z^* \in \mathbb{C} - \overline{f(D_0(z_0, \epsilon))}$, 使得
    \[ D(z^*, \delta) \cap \overline{f(D_0(z_0, \epsilon))}  = \varnothing \]
    令 \[g(z) = \frac{1}{f(z) - z^*}\] 则 \[|g(z)| = \frac{1}{|f(z) - z^*|} \leq \frac{1}{\delta}\] 
    因此 $g(z)$ 在 $z_0$ 的邻域上有界, 由上面定理得 $z_0$ 是 $g(z)$ 的可去奇点, 而由于
    \[f(z) = z^* + \frac{1}{g(z)}\] 所以 $z_0$ 只能是 $f(z)$ 的可去奇点或者极点, 矛盾!
\end{proof}

\section{留数}

\begin{definition}[留数]
    设 $f(z)$ 在 $D_0(z_0, R)$ 内解析, 即 $z_0$ 是 $f(z)$ 的一个孤立奇点, 函数 $f(z)$ 在 $z_0$ 处的\textbf{留数}, 记作 $\mathrm{Res}(f, z_0)$, 定义为
    \[\mathrm{Res}(f, z_0) = \frac{1}{2\pi \mathrm{i}} \int_{|z-z_0|=\rho} f(z) \mathrm{d} z\]
    其中 $0 < \rho < R$.

    当 $\infty$ 为 $f(z)$ 的孤立奇点, 即存在 $R>0$, 使得 $f(z)$ 在 $\mathbb{C}-\overline{D(0,R)}$ 上解析, 则 $f(z)$ 在 $\infty$ 处的留数记作 $\mathrm{Res}(f, \infty)$, 定义为
    \[\mathrm{Res}(f, \infty) = - \frac{1}{2\pi \mathrm{i}} \int_{|z|=\rho} f(z) \mathrm{d}z\]
    其中 $R < \rho < +\infty$
\end{definition}

\begin{thm}[留数定理]
    设 $\varOmega$ 是 $\overline{\mathbb{C}}$ 中以有限条逐段光滑曲线为边界的区域, $\infty \notin \partial \varOmega$, $z_1, \cdots, z_n$ 位于 $\varOmega$ 的内部, 
    再设 $f(z)$ 在 $\varOmega$ 内除去 $z_1, \cdots, z_n$ 外解析, 在 $\overline{\varOmega}$ 上除去 $z_1, \cdots, z_n$ 外连续, 则 
    \[\int_{\partial \varOmega} f(z) \mathrm{d}z = 2 \pi \mathrm{i} \sum\limits_{k=1}^n \mathrm{Res}(f, z_k)\]
\end{thm}

\section{共形映射}

\begin{thm}[单位圆盘自同胚]
    如果 $g: D(0,1) \to D(0,1)$ 是单位圆盘到自身的解析自同胚, 则 $g(z)$ 为如下形式的分式线性变换
    \[g(z) = \mathrm{e}^{\mathrm{i} \theta} \frac{z - z_0}{1 - \overline{z_0}z}\]
    其中 $z_0 \in D(0,1), \theta \in [0, 2\pi]$.
\end{thm}

\end{document}